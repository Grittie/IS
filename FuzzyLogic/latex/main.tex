\documentclass[a4paper,11pt]{article}

\usepackage{geometry}
\geometry{margin=2.5cm}
\usepackage{graphicx}
\usepackage{hyperref}
\usepackage{amsmath}
\usepackage{listings}
\usepackage{color}
\usepackage{caption}
\usepackage{float}

\definecolor{codegray} {gray}{0.95}

\lstset{
    backgroundcolor=\color{codegray},
    basicstyle=\ttfamily\small,
    breaklines=true,
    columns=fullflexible,
}

\title{Intelligent Systems - Fuzzy Logic}
\author{Lars Grit}
\date{\today}

\begin{document}
\maketitle

\section*{Abstract}
Fuzzy logic biedt een manier om menselijke redenering te modelleren in systemen waar onzekerheid en ruis bestaat. Met Python-gebaseerd fuzzy logic framework (\texttt{fuzzylogic}) en een voorbeeld uit Negnevitsky's \emph{Artificial Intelligence: A Guide to Intelligent Systems}. Bouwen wij het volledige \emph{Spare Parts Service Centre} expertsystem.

\section{De Demo: Implementatie van Hoofdstuk 4.7}
In onderstaande code implementeren we volledig het fuzzy expert system uit Sectie 4.7. De inputs zijn genormaliseerd naar het interval $[0,1]$ en de rule-base bestaat uit 27 regels zoals beschreven in de literatuur.

\subsection{Code}
\begin{lstlisting}[language=Python]
from fuzzylogic.classes import Domain, Set, Rule
from fuzzylogic.functions import S, R  # S = left shoulder, R = right shoulder (as in your example)
from fuzzylogic.hedges import very
from typing import Any


mean_delay = Domain("Mean delay m", 0.0, 1.0, res=0.001)            # VS, S, M
servers    = Domain("Number of servers s", 0.0, 1.0, res=0.001)     # S, M, L
util       = Domain("Utilisation factor ρ", 0.0, 1.0, res=0.001)    # L, M, H
spares     = Domain("Number of spares n", 0.0, 1.0, res=0.001)      # VS, S, RS, M, RL, L, VL


def _to_set(x: Any) -> Set:
    return x if isinstance(x, Set) else Set(x)

def middle(left: Any, right: Any) -> Set:
    return _to_set(left) & _to_set(right)

mean_delay.VS = S(0.0, 0.3)
mean_delay.M  = R(0.4, 0.7)
mean_delay.S  = middle(R(0.1, 0.3), S(0.3, 0.5))

servers.S = S(0.0, 0.35)
servers.L = R(0.60, 1.0)
servers.M = middle(R(0.30, 0.50), S(0.50, 0.70))

util.L = S(0.0, 0.6)
util.H = R(0.6, 1.0)
util.M = middle(R(0.4, 0.6), S(0.6, 0.8))

spares.VS = S(0.0, 0.30)
spares.VL = R(0.70, 1.0)
spares.S  = S(0.0, 0.40)
spares.L  = R(0.60, 1.0)
spares.RS = middle(R(0.25, 0.35), S(0.35, 0.45))
spares.M  = middle(R(0.30, 0.50), S(0.50, 0.70))
spares.RL = middle(R(0.55, 0.65), S(0.65, 0.75))


rules = Rule({
    (mean_delay.VS, servers.S, util.L): spares.VS,
    (mean_delay.S,  servers.S, util.L): spares.VS,
    (mean_delay.M,  servers.S, util.L): spares.VS,

    (mean_delay.VS, servers.M, util.L): spares.VS,
    (mean_delay.S,  servers.M, util.L): spares.VS,
    (mean_delay.M,  servers.M, util.L): spares.VS,

    (mean_delay.VS, servers.L, util.L): spares.S,
    (mean_delay.S,  servers.L, util.L): spares.S,
    (mean_delay.M,  servers.L, util.L): spares.VS,

    (mean_delay.VS, servers.S, util.M): spares.S,
    (mean_delay.S,  servers.S, util.M): spares.VS,
    (mean_delay.M,  servers.S, util.M): spares.VS,

    (mean_delay.VS, servers.M, util.M): spares.RS,
    (mean_delay.S,  servers.M, util.M): spares.S,
    (mean_delay.M,  servers.M, util.M): spares.VS,

    (mean_delay.VS, servers.L, util.M): spares.M,
    (mean_delay.S,  servers.L, util.M): spares.RS,
    (mean_delay.M,  servers.L, util.M): spares.S,

    (mean_delay.VS, servers.S, util.H): spares.VL,
    (mean_delay.S,  servers.S, util.H): spares.L,
    (mean_delay.M,  servers.S, util.H): spares.M,

    (mean_delay.VS, servers.M, util.H): spares.M,
    (mean_delay.S,  servers.M, util.H): spares.M,
    (mean_delay.M,  servers.M, util.H): spares.S,

    (mean_delay.VS, servers.L, util.H): spares.RL,
    (mean_delay.S,  servers.L, util.H): spares.M,
    (mean_delay.M,  servers.L, util.H): spares.RS,
})

if __name__ == "__main__":

    values = {
        mean_delay: 0.25,
        servers:    0.50,
        util:       0.70,
    }

    result = rules(values)
    print(f"Recommended number of spares: {result:.3f}")

\end{lstlisting}

\subsection{Resultaat}
Bij de voorbeeldinput geeft het systeem:
\begin{quote}
\texttt{Recommended normalised spares: 0.307}
\end{quote}

Dit komt overeen met een lage, maar niet minimale voorraadniveaus, precies zoals beschreven in het oorspronkelijke hoofdstuk.


\section{Evaluatie van het framework}

\subsection{Is het simpel te gebruiken?}
Ja. Met het framework kan je makkelijk domains, fuzzy sets en rules definiëren. De terminologie volgt ook de fuzzy literatuur.

\subsection{Is de code begrijpelijk?}
De structuur is duidelijk:
\begin{enumerate}
    \item Definieer de domains
    \item Definieer de membership functions
    \item Definieer de rules
    \item Voer een defuzzificatie uit
\end{enumerate}

\subsection{Is het snel?}
Ja. De inference werkt met max-min aggregatie en numerieke intergratie voor defuzz.

\subsection{Is de onderliggende code begrijpelijk ?}
Ja en nee. Het gebruik van \texttt{S}-, \texttt{R}-functies en \texttt{very()}-hedges volgt traditionele fuzzy logic literatuur. Wel zijn alle functions recursive closures, de main classes gebruiken veel dunder methoden.

\subsection{Is grafische output mogelijk?}
Niet ingebouwd, en ook niet gebruikt maar wel mogelijk met matplotlib of andere plotting libraries.

\subsection{Is het veilig voor robotica?}
Voor niet-safety-critical toepassingen wel. De code is niet geoptimaliseerd voor real-time performance.

\end{document}
